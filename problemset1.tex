\PassOptionsToPackage{unicode=true}{hyperref} % options for packages loaded elsewhere
\PassOptionsToPackage{hyphens}{url}
%
\documentclass[
]{article}
\usepackage{lmodern}
\usepackage{amssymb,amsmath}
\usepackage{ifxetex,ifluatex}
\ifnum 0\ifxetex 1\fi\ifluatex 1\fi=0 % if pdftex
  \usepackage[T1]{fontenc}
  \usepackage[utf8]{inputenc}
  \usepackage{textcomp} % provides euro and other symbols
\else % if luatex or xelatex
  \usepackage{unicode-math}
  \defaultfontfeatures{Scale=MatchLowercase}
  \defaultfontfeatures[\rmfamily]{Ligatures=TeX,Scale=1}
\fi
% use upquote if available, for straight quotes in verbatim environments
\IfFileExists{upquote.sty}{\usepackage{upquote}}{}
\IfFileExists{microtype.sty}{% use microtype if available
  \usepackage[]{microtype}
  \UseMicrotypeSet[protrusion]{basicmath} % disable protrusion for tt fonts
}{}
\makeatletter
\@ifundefined{KOMAClassName}{% if non-KOMA class
  \IfFileExists{parskip.sty}{%
    \usepackage{parskip}
  }{% else
    \setlength{\parindent}{0pt}
    \setlength{\parskip}{6pt plus 2pt minus 1pt}}
}{% if KOMA class
  \KOMAoptions{parskip=half}}
\makeatother
\usepackage{xcolor}
\IfFileExists{xurl.sty}{\usepackage{xurl}}{} % add URL line breaks if available
\IfFileExists{bookmark.sty}{\usepackage{bookmark}}{\usepackage{hyperref}}
\hypersetup{
  pdfborder={0 0 0},
  breaklinks=true}
\urlstyle{same}  % don't use monospace font for urls
\usepackage[margin=1in]{geometry}
\usepackage{color}
\usepackage{fancyvrb}
\newcommand{\VerbBar}{|}
\newcommand{\VERB}{\Verb[commandchars=\\\{\}]}
\DefineVerbatimEnvironment{Highlighting}{Verbatim}{commandchars=\\\{\}}
% Add ',fontsize=\small' for more characters per line
\usepackage{framed}
\definecolor{shadecolor}{RGB}{248,248,248}
\newenvironment{Shaded}{\begin{snugshade}}{\end{snugshade}}
\newcommand{\AlertTok}[1]{\textcolor[rgb]{0.94,0.16,0.16}{#1}}
\newcommand{\AnnotationTok}[1]{\textcolor[rgb]{0.56,0.35,0.01}{\textbf{\textit{#1}}}}
\newcommand{\AttributeTok}[1]{\textcolor[rgb]{0.77,0.63,0.00}{#1}}
\newcommand{\BaseNTok}[1]{\textcolor[rgb]{0.00,0.00,0.81}{#1}}
\newcommand{\BuiltInTok}[1]{#1}
\newcommand{\CharTok}[1]{\textcolor[rgb]{0.31,0.60,0.02}{#1}}
\newcommand{\CommentTok}[1]{\textcolor[rgb]{0.56,0.35,0.01}{\textit{#1}}}
\newcommand{\CommentVarTok}[1]{\textcolor[rgb]{0.56,0.35,0.01}{\textbf{\textit{#1}}}}
\newcommand{\ConstantTok}[1]{\textcolor[rgb]{0.00,0.00,0.00}{#1}}
\newcommand{\ControlFlowTok}[1]{\textcolor[rgb]{0.13,0.29,0.53}{\textbf{#1}}}
\newcommand{\DataTypeTok}[1]{\textcolor[rgb]{0.13,0.29,0.53}{#1}}
\newcommand{\DecValTok}[1]{\textcolor[rgb]{0.00,0.00,0.81}{#1}}
\newcommand{\DocumentationTok}[1]{\textcolor[rgb]{0.56,0.35,0.01}{\textbf{\textit{#1}}}}
\newcommand{\ErrorTok}[1]{\textcolor[rgb]{0.64,0.00,0.00}{\textbf{#1}}}
\newcommand{\ExtensionTok}[1]{#1}
\newcommand{\FloatTok}[1]{\textcolor[rgb]{0.00,0.00,0.81}{#1}}
\newcommand{\FunctionTok}[1]{\textcolor[rgb]{0.00,0.00,0.00}{#1}}
\newcommand{\ImportTok}[1]{#1}
\newcommand{\InformationTok}[1]{\textcolor[rgb]{0.56,0.35,0.01}{\textbf{\textit{#1}}}}
\newcommand{\KeywordTok}[1]{\textcolor[rgb]{0.13,0.29,0.53}{\textbf{#1}}}
\newcommand{\NormalTok}[1]{#1}
\newcommand{\OperatorTok}[1]{\textcolor[rgb]{0.81,0.36,0.00}{\textbf{#1}}}
\newcommand{\OtherTok}[1]{\textcolor[rgb]{0.56,0.35,0.01}{#1}}
\newcommand{\PreprocessorTok}[1]{\textcolor[rgb]{0.56,0.35,0.01}{\textit{#1}}}
\newcommand{\RegionMarkerTok}[1]{#1}
\newcommand{\SpecialCharTok}[1]{\textcolor[rgb]{0.00,0.00,0.00}{#1}}
\newcommand{\SpecialStringTok}[1]{\textcolor[rgb]{0.31,0.60,0.02}{#1}}
\newcommand{\StringTok}[1]{\textcolor[rgb]{0.31,0.60,0.02}{#1}}
\newcommand{\VariableTok}[1]{\textcolor[rgb]{0.00,0.00,0.00}{#1}}
\newcommand{\VerbatimStringTok}[1]{\textcolor[rgb]{0.31,0.60,0.02}{#1}}
\newcommand{\WarningTok}[1]{\textcolor[rgb]{0.56,0.35,0.01}{\textbf{\textit{#1}}}}
\usepackage{graphicx,grffile}
\makeatletter
\def\maxwidth{\ifdim\Gin@nat@width>\linewidth\linewidth\else\Gin@nat@width\fi}
\def\maxheight{\ifdim\Gin@nat@height>\textheight\textheight\else\Gin@nat@height\fi}
\makeatother
% Scale images if necessary, so that they will not overflow the page
% margins by default, and it is still possible to overwrite the defaults
% using explicit options in \includegraphics[width, height, ...]{}
\setkeys{Gin}{width=\maxwidth,height=\maxheight,keepaspectratio}
\setlength{\emergencystretch}{3em}  % prevent overfull lines
\providecommand{\tightlist}{%
  \setlength{\itemsep}{0pt}\setlength{\parskip}{0pt}}
\setcounter{secnumdepth}{-2}
% Redefines (sub)paragraphs to behave more like sections
\ifx\paragraph\undefined\else
  \let\oldparagraph\paragraph
  \renewcommand{\paragraph}[1]{\oldparagraph{#1}\mbox{}}
\fi
\ifx\subparagraph\undefined\else
  \let\oldsubparagraph\subparagraph
  \renewcommand{\subparagraph}[1]{\oldsubparagraph{#1}\mbox{}}
\fi

% set default figure placement to htbp
\makeatletter
\def\fps@figure{htbp}
\makeatother


\author{}
\date{\vspace{-2.5em}}

\begin{document}

\#Problem Set 1: Solutions and Analysis using R\\
\#\#R edit by RJ Neel\\
\#\#Original Author: Ben Lambert

\#\#\#Getting started with R\\
R is a free programming language used for statistical analysis (and much
more). Other softwares used for statistical analysis are Gretl, IBM
SPSS, MATLAB, Python and many more. Python is the leading data analysis
language. R is popular among academics.

The following link covers all the necessary information regarding R and
R studio installation
Link:\url{https://www.youtube.com/watch?v=9-RrkJQQYqY}

\#\#\#The Crime-Unemployment data set

\textbf{\#Questions 1 to 2: Histograms}\\
Let's begin by Visualizing the data.We do this by plotting Histograms.

\begin{verbatim}
##        State Unemployment Violence
## 1    Alabama          7.2    383.7
## 2     Alaska          7.0    635.3
## 3    Arizona          7.7    413.6
## 4   Arkansas          7.2    503.5
## 5 California          8.9    439.6
## 6   Colorado          6.9    323.7
\end{verbatim}

\includegraphics{problemset1_files/figure-latex/unnamed-chunk-1-1.pdf}
\includegraphics{problemset1_files/figure-latex/unnamed-chunk-1-2.pdf}

\textbf{\#Questions 3: Maximum Violence state}\\
The State with Maximum violence is

\begin{verbatim}
## 
## Attaching package: 'dplyr'
\end{verbatim}

\begin{verbatim}
## The following objects are masked from 'package:stats':
## 
##     filter, lag
\end{verbatim}

\begin{verbatim}
## The following objects are masked from 'package:base':
## 
##     intersect, setdiff, setequal, union
\end{verbatim}

\begin{verbatim}
##                  State Unemployment Violence
## 9 District of Columbia          8.5   1326.8
\end{verbatim}

District of Columbia

\textbf{\#Questions 4: Summary of Data}\\
Summary Statistics

\begin{Shaded}
\begin{Highlighting}[]
\KeywordTok{summary}\NormalTok{(data)}
\end{Highlighting}
\end{Shaded}

\begin{verbatim}
##     State            Unemployment      Violence     
##  Length:51          Min.   :2.900   Min.   : 122.1  
##  Class :character   1st Qu.:5.550   1st Qu.: 260.9  
##  Mode  :character   Median :6.900   Median : 323.7  
##                     Mean   :6.765   Mean   : 385.3  
##                     3rd Qu.:7.800   3rd Qu.: 475.1  
##                     Max.   :9.600   Max.   :1326.8
\end{verbatim}

\textbf{\#Questions 5: Relationship between Violence and Unemployment}\\
Let us try to check this using scatterplot\\
\includegraphics{problemset1_files/figure-latex/unnamed-chunk-4-1.pdf}
\includegraphics{problemset1_files/figure-latex/unnamed-chunk-4-2.pdf}

\emph{It appears to have a \textbf{positive} relation}

\textbf{\#Questions 6: Correlation}\\
The Correlation between Violence and unemployment is 0.42

\textbf{\#Questions 7 to 8: Regression}

\begin{verbatim}
## 
## Call:
## lm(formula = data$Violence ~ data$Unemployment)
## 
## Residuals:
##     Min      1Q  Median      3Q     Max 
## -263.13 -110.84  -40.78   87.49  848.95 
## 
## Coefficients:
##                   Estimate Std. Error t value Pr(>|t|)   
## (Intercept)          24.40     113.90   0.214   0.8313   
## data$Unemployment    53.35      16.43   3.248   0.0021 **
## ---
## Signif. codes:  0 '***' 0.001 '**' 0.01 '*' 0.05 '.' 0.1 ' ' 1
## 
## Residual standard error: 178.8 on 49 degrees of freedom
## Multiple R-squared:  0.1771, Adjusted R-squared:  0.1603 
## F-statistic: 10.55 on 1 and 49 DF,  p-value: 0.002102
\end{verbatim}

\begin{verbatim}
##       (Intercept) data$Unemployment 
##          24.39789          53.34785
\end{verbatim}

\textbf{\#Questions 9: Interpretation of Beta}\\
For every 1 percentage change in unemployment we can expect an increase
of 53 crimes per 100,000.\\
\textbf{\#Questions 10: Regression of Unemployment (Dependent) on
Violence (Independent) }

\begin{verbatim}
## 
## Call:
## lm(formula = data$Unemployment ~ data$Violence)
## 
## Residuals:
##     Min      1Q  Median      3Q     Max 
## -3.3474 -0.9712  0.0427  1.0554  2.9599 
## 
## Coefficients:
##               Estimate Std. Error t value Pr(>|t|)    
## (Intercept)   5.485383   0.440645  12.449   <2e-16 ***
## data$Violence 0.003320   0.001022   3.248   0.0021 ** 
## ---
## Signif. codes:  0 '***' 0.001 '**' 0.01 '*' 0.05 '.' 0.1 ' ' 1
## 
## Residual standard error: 1.411 on 49 degrees of freedom
## Multiple R-squared:  0.1771, Adjusted R-squared:  0.1603 
## F-statistic: 10.55 on 1 and 49 DF,  p-value: 0.002102
\end{verbatim}

\begin{verbatim}
##   (Intercept) data$Violence 
##   5.485382917   0.003320499
\end{verbatim}

\textbf{\#Questions 11: Interpretation of Beta}\\
For every 1 unit increase in violence per 100,00 we can expect an
increase of 0.003 percentage unit in Unemployment.

\textbf{\#Questions 11: Interpretation of Beta}\\
Regression does not indicate causality but on average we can expect
higher unemployment to be associated with high violence. Also, Higher
violence may itself lead to unemployment.

\#Other Analysis

\begin{verbatim}
## Warning: Use of `data$Unemployment` is discouraged. Use `Unemployment` instead.
\end{verbatim}

\begin{verbatim}
## Warning: Use of `data$Violence` is discouraged. Use `Violence` instead.
\end{verbatim}

\begin{verbatim}
## Warning: Continuous x aesthetic -- did you forget aes(group=...)?
\end{verbatim}

\includegraphics{problemset1_files/figure-latex/unnamed-chunk-7-1.pdf}
\includegraphics{problemset1_files/figure-latex/unnamed-chunk-7-2.pdf}
\includegraphics{problemset1_files/figure-latex/unnamed-chunk-7-3.pdf}
\includegraphics{problemset1_files/figure-latex/unnamed-chunk-7-4.pdf}

\end{document}
