\PassOptionsToPackage{unicode=true}{hyperref} % options for packages loaded elsewhere
\PassOptionsToPackage{hyphens}{url}
%
\documentclass[
]{article}
\usepackage{lmodern}
\usepackage{amssymb,amsmath}
\usepackage{ifxetex,ifluatex}
\ifnum 0\ifxetex 1\fi\ifluatex 1\fi=0 % if pdftex
  \usepackage[T1]{fontenc}
  \usepackage[utf8]{inputenc}
  \usepackage{textcomp} % provides euro and other symbols
\else % if luatex or xelatex
  \usepackage{unicode-math}
  \defaultfontfeatures{Scale=MatchLowercase}
  \defaultfontfeatures[\rmfamily]{Ligatures=TeX,Scale=1}
\fi
% use upquote if available, for straight quotes in verbatim environments
\IfFileExists{upquote.sty}{\usepackage{upquote}}{}
\IfFileExists{microtype.sty}{% use microtype if available
  \usepackage[]{microtype}
  \UseMicrotypeSet[protrusion]{basicmath} % disable protrusion for tt fonts
}{}
\makeatletter
\@ifundefined{KOMAClassName}{% if non-KOMA class
  \IfFileExists{parskip.sty}{%
    \usepackage{parskip}
  }{% else
    \setlength{\parindent}{0pt}
    \setlength{\parskip}{6pt plus 2pt minus 1pt}}
}{% if KOMA class
  \KOMAoptions{parskip=half}}
\makeatother
\usepackage{xcolor}
\IfFileExists{xurl.sty}{\usepackage{xurl}}{} % add URL line breaks if available
\IfFileExists{bookmark.sty}{\usepackage{bookmark}}{\usepackage{hyperref}}
\hypersetup{
  pdfborder={0 0 0},
  breaklinks=true}
\urlstyle{same}  % don't use monospace font for urls
\usepackage[margin=1in]{geometry}
\usepackage{color}
\usepackage{fancyvrb}
\newcommand{\VerbBar}{|}
\newcommand{\VERB}{\Verb[commandchars=\\\{\}]}
\DefineVerbatimEnvironment{Highlighting}{Verbatim}{commandchars=\\\{\}}
% Add ',fontsize=\small' for more characters per line
\usepackage{framed}
\definecolor{shadecolor}{RGB}{248,248,248}
\newenvironment{Shaded}{\begin{snugshade}}{\end{snugshade}}
\newcommand{\AlertTok}[1]{\textcolor[rgb]{0.94,0.16,0.16}{#1}}
\newcommand{\AnnotationTok}[1]{\textcolor[rgb]{0.56,0.35,0.01}{\textbf{\textit{#1}}}}
\newcommand{\AttributeTok}[1]{\textcolor[rgb]{0.77,0.63,0.00}{#1}}
\newcommand{\BaseNTok}[1]{\textcolor[rgb]{0.00,0.00,0.81}{#1}}
\newcommand{\BuiltInTok}[1]{#1}
\newcommand{\CharTok}[1]{\textcolor[rgb]{0.31,0.60,0.02}{#1}}
\newcommand{\CommentTok}[1]{\textcolor[rgb]{0.56,0.35,0.01}{\textit{#1}}}
\newcommand{\CommentVarTok}[1]{\textcolor[rgb]{0.56,0.35,0.01}{\textbf{\textit{#1}}}}
\newcommand{\ConstantTok}[1]{\textcolor[rgb]{0.00,0.00,0.00}{#1}}
\newcommand{\ControlFlowTok}[1]{\textcolor[rgb]{0.13,0.29,0.53}{\textbf{#1}}}
\newcommand{\DataTypeTok}[1]{\textcolor[rgb]{0.13,0.29,0.53}{#1}}
\newcommand{\DecValTok}[1]{\textcolor[rgb]{0.00,0.00,0.81}{#1}}
\newcommand{\DocumentationTok}[1]{\textcolor[rgb]{0.56,0.35,0.01}{\textbf{\textit{#1}}}}
\newcommand{\ErrorTok}[1]{\textcolor[rgb]{0.64,0.00,0.00}{\textbf{#1}}}
\newcommand{\ExtensionTok}[1]{#1}
\newcommand{\FloatTok}[1]{\textcolor[rgb]{0.00,0.00,0.81}{#1}}
\newcommand{\FunctionTok}[1]{\textcolor[rgb]{0.00,0.00,0.00}{#1}}
\newcommand{\ImportTok}[1]{#1}
\newcommand{\InformationTok}[1]{\textcolor[rgb]{0.56,0.35,0.01}{\textbf{\textit{#1}}}}
\newcommand{\KeywordTok}[1]{\textcolor[rgb]{0.13,0.29,0.53}{\textbf{#1}}}
\newcommand{\NormalTok}[1]{#1}
\newcommand{\OperatorTok}[1]{\textcolor[rgb]{0.81,0.36,0.00}{\textbf{#1}}}
\newcommand{\OtherTok}[1]{\textcolor[rgb]{0.56,0.35,0.01}{#1}}
\newcommand{\PreprocessorTok}[1]{\textcolor[rgb]{0.56,0.35,0.01}{\textit{#1}}}
\newcommand{\RegionMarkerTok}[1]{#1}
\newcommand{\SpecialCharTok}[1]{\textcolor[rgb]{0.00,0.00,0.00}{#1}}
\newcommand{\SpecialStringTok}[1]{\textcolor[rgb]{0.31,0.60,0.02}{#1}}
\newcommand{\StringTok}[1]{\textcolor[rgb]{0.31,0.60,0.02}{#1}}
\newcommand{\VariableTok}[1]{\textcolor[rgb]{0.00,0.00,0.00}{#1}}
\newcommand{\VerbatimStringTok}[1]{\textcolor[rgb]{0.31,0.60,0.02}{#1}}
\newcommand{\WarningTok}[1]{\textcolor[rgb]{0.56,0.35,0.01}{\textbf{\textit{#1}}}}
\usepackage{graphicx,grffile}
\makeatletter
\def\maxwidth{\ifdim\Gin@nat@width>\linewidth\linewidth\else\Gin@nat@width\fi}
\def\maxheight{\ifdim\Gin@nat@height>\textheight\textheight\else\Gin@nat@height\fi}
\makeatother
% Scale images if necessary, so that they will not overflow the page
% margins by default, and it is still possible to overwrite the defaults
% using explicit options in \includegraphics[width, height, ...]{}
\setkeys{Gin}{width=\maxwidth,height=\maxheight,keepaspectratio}
\setlength{\emergencystretch}{3em}  % prevent overfull lines
\providecommand{\tightlist}{%
  \setlength{\itemsep}{0pt}\setlength{\parskip}{0pt}}
\setcounter{secnumdepth}{-2}
% Redefines (sub)paragraphs to behave more like sections
\ifx\paragraph\undefined\else
  \let\oldparagraph\paragraph
  \renewcommand{\paragraph}[1]{\oldparagraph{#1}\mbox{}}
\fi
\ifx\subparagraph\undefined\else
  \let\oldsubparagraph\subparagraph
  \renewcommand{\subparagraph}[1]{\oldsubparagraph{#1}\mbox{}}
\fi

% set default figure placement to htbp
\makeatletter
\def\fps@figure{htbp}
\makeatother


\author{}
\date{\vspace{-2.5em}}

\begin{document}

\hypertarget{introductory-econometrics-chapter-1}{%
\subsection{Introductory Econometrics: Chapter
1}\label{introductory-econometrics-chapter-1}}

\hypertarget{author-j-m-woolridge}{%
\subsection{Author: J M Woolridge}\label{author-j-m-woolridge}}

\hypertarget{r-code-compilation-by-rj-neel}{%
\subsection{R Code Compilation by RJ
Neel}\label{r-code-compilation-by-rj-neel}}

\hypertarget{end-of-chapter-1-exercises-page-39}{%
\section{End of Chapter 1 exercises (Page
39)}\label{end-of-chapter-1-exercises-page-39}}

\hypertarget{computer-exercises}{%
\subsubsection{Computer Exercises}\label{computer-exercises}}

\textbf{C1: Use the data in WAGE1 for this exercise}

\begin{Shaded}
\begin{Highlighting}[]
\CommentTok{## Computer Exercise C1}
\KeywordTok{library}\NormalTok{(wooldridge) }\CommentTok{#load the Woolridge Package}
\NormalTok{wage1}
\NormalTok{?wage1 }\CommentTok{#Description of the dataset}
\end{Highlighting}
\end{Shaded}

\begin{verbatim}
## starting httpd help server ... done
\end{verbatim}

\begin{Shaded}
\begin{Highlighting}[]
\KeywordTok{head}\NormalTok{(wage1) }\CommentTok{#First 6 rows. Easy to view}
\KeywordTok{ncol}\NormalTok{(wage1) }\CommentTok{# No of rows}
\KeywordTok{nrow}\NormalTok{(wage1) }\CommentTok{#No. of columns}
\end{Highlighting}
\end{Shaded}

\textbf{(i) Find the average education level in the sample. What are the
lowest and highest years of education?}

\emph{Solution}

\begin{Shaded}
\begin{Highlighting}[]
\KeywordTok{summary}\NormalTok{(wage1}\OperatorTok{$}\NormalTok{educ)}
\end{Highlighting}
\end{Shaded}

\begin{verbatim}
##    Min. 1st Qu.  Median    Mean 3rd Qu.    Max. 
##    0.00   12.00   12.00   12.56   14.00   18.00
\end{verbatim}

\begin{Shaded}
\begin{Highlighting}[]
\CommentTok{#Alternatively}
\KeywordTok{mean}\NormalTok{(wage1}\OperatorTok{$}\NormalTok{educ) }\CommentTok{#avg education level}
\end{Highlighting}
\end{Shaded}

\begin{verbatim}
## [1] 12.56274
\end{verbatim}

\begin{Shaded}
\begin{Highlighting}[]
\KeywordTok{min}\NormalTok{(wage1}\OperatorTok{$}\NormalTok{educ) }\CommentTok{#min education leve}
\end{Highlighting}
\end{Shaded}

\begin{verbatim}
## [1] 0
\end{verbatim}

\begin{Shaded}
\begin{Highlighting}[]
\KeywordTok{max}\NormalTok{(wage1}\OperatorTok{$}\NormalTok{educ) }\CommentTok{#max}
\end{Highlighting}
\end{Shaded}

\begin{verbatim}
## [1] 18
\end{verbatim}

\textbf{(ii) Find the average hourly wage in the sample. Does it seem
high or low? }

\emph{Solution}

\begin{Shaded}
\begin{Highlighting}[]
\KeywordTok{mean}\NormalTok{(wage1}\OperatorTok{$}\NormalTok{wage) }\CommentTok{#Gives you the average hourly wage }
\end{Highlighting}
\end{Shaded}

\begin{verbatim}
## [1] 5.896103
\end{verbatim}

\begin{Shaded}
\begin{Highlighting}[]
\KeywordTok{summary}\NormalTok{(wage1}\OperatorTok{$}\NormalTok{wage) }\CommentTok{#Wage appears to be low}
\end{Highlighting}
\end{Shaded}

\begin{verbatim}
##    Min. 1st Qu.  Median    Mean 3rd Qu.    Max. 
##   0.530   3.330   4.650   5.896   6.880  24.980
\end{verbatim}

\begin{Shaded}
\begin{Highlighting}[]
\KeywordTok{hist}\NormalTok{(wage1}\OperatorTok{$}\NormalTok{wage) }\CommentTok{#Clearly skewed towards right}
\end{Highlighting}
\end{Shaded}

\includegraphics{woolridge_c1_files/figure-latex/unnamed-chunk-3-1.pdf}

\textbf{(iii) The wage data are reported in 1976 dollars. Using the
Internet or a printed source, find the Consumer Price Index (CPI) for
the years 1976 and 2013.}

\emph{Solution} Using Table B-60 in the 2004 Economic Report of the
President, the CPI was 56.9 in 1976 and 233 in 2013.

\textbf{(iv) Use the CPI values from part (iii) to find the average
hourly wage in 2013 dollars. Now does the average hourly wage seem
reasonable? }

\emph{Solution} To convert 1976 dollars into 2013 dollars, we use the
ratio of the CPIs, which is 233/56.9 ≈ 4.09. Therefore, the average
hourly wage in 2013 dollars is roughly 4.09(\$\textbackslash{}5.90)
≈~\$\textbackslash{}24.13, which is a reasonable figure.

\textbf{(v) How many women are in the sample? How many men?}

\emph{Solution}

\begin{Shaded}
\begin{Highlighting}[]
\KeywordTok{head}\NormalTok{(wage1)}
\end{Highlighting}
\end{Shaded}

\begin{verbatim}
##   wage educ exper tenure nonwhite female married numdep smsa northcen south
## 1 3.10   11     2      0        0      1       0      2    1        0     0
## 2 3.24   12    22      2        0      1       1      3    1        0     0
## 3 3.00   11     2      0        0      0       0      2    0        0     0
## 4 6.00    8    44     28        0      0       1      0    1        0     0
## 5 5.30   12     7      2        0      0       1      1    0        0     0
## 6 8.75   16     9      8        0      0       1      0    1        0     0
##   west construc ndurman trcommpu trade services profserv profocc clerocc
## 1    1        0       0        0     0        0        0       0       0
## 2    1        0       0        0     0        1        0       0       0
## 3    1        0       0        0     1        0        0       0       0
## 4    1        0       0        0     0        0        0       0       1
## 5    1        0       0        0     0        0        0       0       0
## 6    1        0       0        0     0        0        1       1       0
##   servocc    lwage expersq tenursq
## 1       0 1.131402       4       0
## 2       1 1.175573     484       4
## 3       0 1.098612       4       0
## 4       0 1.791759    1936     784
## 5       0 1.667707      49       4
## 6       0 2.169054      81      64
\end{verbatim}

\begin{Shaded}
\begin{Highlighting}[]
\CommentTok{#Notice the female column is a binary variable implying 1 for female and 0 for male requring us to proceed with 'dplyr'}
\KeywordTok{library}\NormalTok{(dplyr)}
\end{Highlighting}
\end{Shaded}

\begin{verbatim}
## 
## Attaching package: 'dplyr'
\end{verbatim}

\begin{verbatim}
## The following objects are masked from 'package:stats':
## 
##     filter, lag
\end{verbatim}

\begin{verbatim}
## The following objects are masked from 'package:base':
## 
##     intersect, setdiff, setequal, union
\end{verbatim}

\begin{Shaded}
\begin{Highlighting}[]
\NormalTok{w=}\KeywordTok{nrow}\NormalTok{(wage1 }\OperatorTok\StringTok{ }\KeywordTok{group_by}\NormalTok{(female) }\OperatorTok\StringTok{ }\KeywordTok{filter}\NormalTok{(female}\OperatorTok{==}\StringTok{'1'}\NormalTok{))}
\NormalTok{w}
\end{Highlighting}
\end{Shaded}

\begin{verbatim}
## [1] 252
\end{verbatim}

\begin{Shaded}
\begin{Highlighting}[]
\NormalTok{m=}\KeywordTok{nrow}\NormalTok{(wage1)}\OperatorTok{-}\NormalTok{w}
\NormalTok{m}
\end{Highlighting}
\end{Shaded}

\begin{verbatim}
## [1] 274
\end{verbatim}

End of Computer Exercise 1

\textbf{C2: Use the data in BWGHT to answer this question}

\begin{enumerate}
\def\labelenumi{(\roman{enumi})}
\tightlist
\item
  How many women are in the sample, and how many report smoking during
  pregnancy?
\end{enumerate}

\emph{Solution}

\begin{Shaded}
\begin{Highlighting}[]
\CommentTok{#Note: This data set contains all women}
\KeywordTok{nrow}\NormalTok{(bwght) }\CommentTok{#No of women smoking}
\end{Highlighting}
\end{Shaded}

\begin{verbatim}
## [1] 1388
\end{verbatim}

\begin{enumerate}
\def\labelenumi{(\roman{enumi})}
\setcounter{enumi}{1}
\tightlist
\item
  What is the average number of cigarettes smoked per day? Is the
  average a good measure of the ``typical'' woman in this case? Explain.
\end{enumerate}

\emph{Solution}

\begin{Shaded}
\begin{Highlighting}[]
\KeywordTok{mean}\NormalTok{(bwght}\OperatorTok{$}\NormalTok{cigs)}
\end{Highlighting}
\end{Shaded}

\begin{verbatim}
## [1] 2.087176
\end{verbatim}

\begin{Shaded}
\begin{Highlighting}[]
\KeywordTok{summary}\NormalTok{(bwght}\OperatorTok{$}\NormalTok{cigs)}
\end{Highlighting}
\end{Shaded}

\begin{verbatim}
##    Min. 1st Qu.  Median    Mean 3rd Qu.    Max. 
##   0.000   0.000   0.000   2.087   0.000  50.000
\end{verbatim}

\begin{Shaded}
\begin{Highlighting}[]
\KeywordTok{hist}\NormalTok{(bwght}\OperatorTok{$}\NormalTok{cigs)}
\end{Highlighting}
\end{Shaded}

\includegraphics{woolridge_c1_files/figure-latex/unnamed-chunk-6-1.pdf}

\begin{Shaded}
\begin{Highlighting}[]
\CommentTok{#Based on the histogram and range it appears to be a good measure.}
\end{Highlighting}
\end{Shaded}

\begin{enumerate}
\def\labelenumi{(\roman{enumi})}
\setcounter{enumi}{2}
\tightlist
\item
  Among women who smoked during pregnancy, what is the average number of
  cigarettes smoked per day? How does this compare with your answer from
  part (ii), and why?
\end{enumerate}

\begin{Shaded}
\begin{Highlighting}[]
\NormalTok{avg_all=}\KeywordTok{mean}\NormalTok{(bwght}\OperatorTok{$}\NormalTok{cigs)}
\NormalTok{avg_all}
\end{Highlighting}
\end{Shaded}

\begin{verbatim}
## [1] 2.087176
\end{verbatim}

\begin{Shaded}
\begin{Highlighting}[]
\KeywordTok{library}\NormalTok{(dplyr)}
\KeywordTok{nrow}\NormalTok{(bwght }\OperatorTok\StringTok{ }\KeywordTok{group_by}\NormalTok{(cigs) }\OperatorTok\StringTok{ }\KeywordTok{filter}\NormalTok{(cigs}\OperatorTok{==}\StringTok{'0'}\NormalTok{))}
\end{Highlighting}
\end{Shaded}

\begin{verbatim}
## [1] 1176
\end{verbatim}

\begin{Shaded}
\begin{Highlighting}[]
\NormalTok{s=(bwght }\OperatorTok\StringTok{ }\KeywordTok{group_by}\NormalTok{(cigs) }\OperatorTok\StringTok{ }\KeywordTok{filter}\NormalTok{(cigs}\OperatorTok{>}\StringTok{'0'}\NormalTok{))}
\NormalTok{avg_s=}\KeywordTok{mean}\NormalTok{(s}\OperatorTok{$}\NormalTok{cigs)}
\NormalTok{avg_s}
\end{Highlighting}
\end{Shaded}

\begin{verbatim}
## [1] 13.66509
\end{verbatim}

\begin{Shaded}
\begin{Highlighting}[]
\KeywordTok{hist}\NormalTok{(s}\OperatorTok{$}\NormalTok{cigs, }\DataTypeTok{xlab=}\StringTok{'Pregnant Women Smokers'}\NormalTok{)}
\end{Highlighting}
\end{Shaded}

\includegraphics{woolridge_c1_files/figure-latex/unnamed-chunk-7-1.pdf}

\begin{Shaded}
\begin{Highlighting}[]
\CommentTok{#It markedly differs from the privious average by about 11 units.}
\end{Highlighting}
\end{Shaded}

\textbf{C3 The data in MEAP01 are for the state of Michigan in the year
2001. Use these data to answer the following questions.}

\begin{Shaded}
\begin{Highlighting}[]
\KeywordTok{head}\NormalTok{(meap01)}
\end{Highlighting}
\end{Shaded}

\begin{verbatim}
##   dcode bcode math4 read4 lunch enroll  expend    exppp  lenroll  lexpend
## 1  1010  4937  83.3  77.8 40.60    468 2747475 5870.673 6.148468 14.82619
## 2  2070   597  90.3  82.3 27.10    679 1505772 2217.632 6.520621 14.22482
## 3  2080  4860  61.9  71.4 41.75    400 2121871 5304.678 5.991465 14.56781
## 4  3010   790  85.7  60.0 12.75    251 1211034 4824.836 5.525453 14.00698
## 5  3010  1403  77.3  59.1 17.08    439 1913501 4358.772 6.084499 14.46445
## 6  3010  4056  85.2  67.0 23.17    561 2637483 4701.396 6.329721 14.78534
##     lexppp
## 1 8.677725
## 2 7.704195
## 3 8.576344
## 4 8.481532
## 5 8.379946
## 6 8.455615
\end{verbatim}

\textbf{(i) Find the largest and smallest values of math4. Does the
range make sense? Explain. }

\emph{Solution}

\begin{Shaded}
\begin{Highlighting}[]
\KeywordTok{head}\NormalTok{(meap01)}
\end{Highlighting}
\end{Shaded}

\begin{verbatim}
##   dcode bcode math4 read4 lunch enroll  expend    exppp  lenroll  lexpend
## 1  1010  4937  83.3  77.8 40.60    468 2747475 5870.673 6.148468 14.82619
## 2  2070   597  90.3  82.3 27.10    679 1505772 2217.632 6.520621 14.22482
## 3  2080  4860  61.9  71.4 41.75    400 2121871 5304.678 5.991465 14.56781
## 4  3010   790  85.7  60.0 12.75    251 1211034 4824.836 5.525453 14.00698
## 5  3010  1403  77.3  59.1 17.08    439 1913501 4358.772 6.084499 14.46445
## 6  3010  4056  85.2  67.0 23.17    561 2637483 4701.396 6.329721 14.78534
##     lexppp
## 1 8.677725
## 2 7.704195
## 3 8.576344
## 4 8.481532
## 5 8.379946
## 6 8.455615
\end{verbatim}

\begin{Shaded}
\begin{Highlighting}[]
\KeywordTok{summary}\NormalTok{(meap01}\OperatorTok{$}\NormalTok{math4) }\CommentTok{# It makes sense as percentage is between 0 and 100}
\end{Highlighting}
\end{Shaded}

\begin{verbatim}
##    Min. 1st Qu.  Median    Mean 3rd Qu.    Max. 
##    0.00   61.60   76.40   71.91   87.00  100.00
\end{verbatim}

\textbf{(ii) How many schools have a perfect pass rate on the math test?
What percentage is this of the total sample? }

\emph{Solution}

\begin{Shaded}
\begin{Highlighting}[]
\KeywordTok{library}\NormalTok{(dplyr)}
\NormalTok{passrate100=}\KeywordTok{nrow}\NormalTok{(meap01 }\OperatorTok\StringTok{ }\KeywordTok{group_by}\NormalTok{(math4) }\OperatorTok\StringTok{ }\KeywordTok{filter}\NormalTok{(math4}\OperatorTok{==}\StringTok{'100'}\NormalTok{))}
\NormalTok{passrate100}
\end{Highlighting}
\end{Shaded}

\begin{verbatim}
## [1] 38
\end{verbatim}

\begin{Shaded}
\begin{Highlighting}[]
\NormalTok{samplesize=}\KeywordTok{nrow}\NormalTok{(meap01)}
\NormalTok{samplesize}
\end{Highlighting}
\end{Shaded}

\begin{verbatim}
## [1] 1823
\end{verbatim}

\begin{Shaded}
\begin{Highlighting}[]
\NormalTok{percent_passrate=}\KeywordTok{round}\NormalTok{((passrate100}\OperatorTok{/}\NormalTok{samplesize)}\OperatorTok{*}\DecValTok{100}\NormalTok{,}\DecValTok{2}\NormalTok{)}
\NormalTok{percent_passrate}
\end{Highlighting}
\end{Shaded}

\begin{verbatim}
## [1] 2.08
\end{verbatim}

\textbf{(iii) How many schools have math pass rates of exactly 50\%? }

\emph{Solution}

\begin{Shaded}
\begin{Highlighting}[]
\KeywordTok{library}\NormalTok{(dplyr)}

\KeywordTok{nrow}\NormalTok{(meap01 }\OperatorTok\StringTok{ }\KeywordTok{group_by}\NormalTok{(math4) }\OperatorTok\StringTok{ }\KeywordTok{filter}\NormalTok{(math4}\OperatorTok{==}\StringTok{'50'}\NormalTok{))}
\end{Highlighting}
\end{Shaded}

\begin{verbatim}
## [1] 17
\end{verbatim}

\textbf{(iv) Compare the average pass rates for the math and reading
scores. Which test is harder to pass? }

\emph{Solution}

\begin{Shaded}
\begin{Highlighting}[]
\NormalTok{pass_m=}\KeywordTok{mean}\NormalTok{(meap01}\OperatorTok{$}\NormalTok{math4)}
\NormalTok{pass_m}
\end{Highlighting}
\end{Shaded}

\begin{verbatim}
## [1] 71.909
\end{verbatim}

\begin{Shaded}
\begin{Highlighting}[]
\NormalTok{pass_r=}\KeywordTok{mean}\NormalTok{(meap01}\OperatorTok{$}\NormalTok{read4)}
\NormalTok{pass_r}
\end{Highlighting}
\end{Shaded}

\begin{verbatim}
## [1] 60.06188
\end{verbatim}

\begin{Shaded}
\begin{Highlighting}[]
\CommentTok{#Cleary Reading is much more difficult to pass}
\end{Highlighting}
\end{Shaded}

\textbf{(v) Find the correlation between math4 and read4. What do you
conclude? }

\emph{Solution}

\begin{Shaded}
\begin{Highlighting}[]
\KeywordTok{cor}\NormalTok{(meap01}\OperatorTok{$}\NormalTok{math4,meap01}\OperatorTok{$}\NormalTok{read4)}
\end{Highlighting}
\end{Shaded}

\begin{verbatim}
## [1] 0.8427281
\end{verbatim}

\begin{Shaded}
\begin{Highlighting}[]
\KeywordTok{library}\NormalTok{(ggplot2)}
\KeywordTok{ggplot}\NormalTok{(}\DataTypeTok{data=}\NormalTok{meap01,}\KeywordTok{aes}\NormalTok{(}\DataTypeTok{x=}\NormalTok{meap01}\OperatorTok{$}\NormalTok{math4,}\DataTypeTok{y=}\NormalTok{meap01}\OperatorTok{$}\NormalTok{read4))}\OperatorTok{+}\KeywordTok{geom_point}\NormalTok{(}\DataTypeTok{col=}\StringTok{'salmon'}\NormalTok{)}\OperatorTok{+}\KeywordTok{ggtitle}\NormalTok{(}\StringTok{"Scatterplot: Maths vs Reading"}\NormalTok{)}\OperatorTok{+}\KeywordTok{xlab}\NormalTok{(}\StringTok{"Math Score"}\NormalTok{)}\OperatorTok{+}\KeywordTok{ylab}\NormalTok{(}\StringTok{"Reading Score"}\NormalTok{)}
\end{Highlighting}
\end{Shaded}

\includegraphics{woolridge_c1_files/figure-latex/unnamed-chunk-13-1.pdf}

\begin{Shaded}
\begin{Highlighting}[]
\CommentTok{#It is strongly positive}
\end{Highlighting}
\end{Shaded}

\textbf{(vi) The variable exppp is expenditure per pupil. Find the
average of exppp along with its standard deviation. Would you say there
is wide variation in per pupil spending? }

\begin{Shaded}
\begin{Highlighting}[]
\KeywordTok{mean}\NormalTok{(meap01}\OperatorTok{$}\NormalTok{exppp)}
\end{Highlighting}
\end{Shaded}

\begin{verbatim}
## [1] 5194.865
\end{verbatim}

\begin{Shaded}
\begin{Highlighting}[]
\KeywordTok{sd}\NormalTok{(meap01}\OperatorTok{$}\NormalTok{exppp)}
\end{Highlighting}
\end{Shaded}

\begin{verbatim}
## [1] 1091.89
\end{verbatim}

\begin{Shaded}
\begin{Highlighting}[]
\KeywordTok{summary}\NormalTok{(meap01}\OperatorTok{$}\NormalTok{exppp)}
\end{Highlighting}
\end{Shaded}

\begin{verbatim}
##    Min. 1st Qu.  Median    Mean 3rd Qu.    Max. 
##    1207    4502    5078    5195    5767   11958
\end{verbatim}

\begin{Shaded}
\begin{Highlighting}[]
\KeywordTok{library}\NormalTok{(ggplot2)}
\KeywordTok{ggplot}\NormalTok{(}\DataTypeTok{data=}\NormalTok{meap01,}\KeywordTok{aes}\NormalTok{(}\DataTypeTok{x=}\NormalTok{meap01}\OperatorTok{$}\NormalTok{exppp))}\OperatorTok{+}\KeywordTok{geom_histogram}\NormalTok{(}\DataTypeTok{col=}\StringTok{'dark grey'}\NormalTok{,}\DataTypeTok{fill=}\StringTok{'white'}\NormalTok{)}\OperatorTok{+}\KeywordTok{ggtitle}\NormalTok{(}\StringTok{"Histogram: Expenditure per Pupil"}\NormalTok{)}\OperatorTok{+}\KeywordTok{xlab}\NormalTok{(}\StringTok{"Expenditure"}\NormalTok{)}\OperatorTok{+}\KeywordTok{ylab}\NormalTok{(}\StringTok{"Frequency"}\NormalTok{)}
\end{Highlighting}
\end{Shaded}

\begin{verbatim}
## `stat_bin()` using `bins = 30`. Pick better value with `binwidth`.
\end{verbatim}

\includegraphics{woolridge_c1_files/figure-latex/unnamed-chunk-14-1.pdf}

\begin{Shaded}
\begin{Highlighting}[]
\CommentTok{#Considering Min=1207 and Max=11958, It is significantly wide}
\end{Highlighting}
\end{Shaded}

\textbf{(vii) Suppose School A spends \$6,000 per student and School B
spends \$5,500 per student. By what percentage does School A's spending
exceed School B's? Compare this to 100 · {[}log(6,000) -- log(5,500){]},
which is the approximation percentage difference based on the difference
in the natural logs. (See Section A.4 in Appendix A.)}

\emph{Solution}

\begin{Shaded}
\begin{Highlighting}[]
\KeywordTok{round}\NormalTok{((}\KeywordTok{log}\NormalTok{(}\DecValTok{6000}\NormalTok{)}\OperatorTok{-}\KeywordTok{log}\NormalTok{(}\DecValTok{5500}\NormalTok{))}\OperatorTok{*}\DecValTok{100}\NormalTok{,}\DecValTok{2}\NormalTok{) }\CommentTok{# Gives the Percentage}
\end{Highlighting}
\end{Shaded}

\begin{verbatim}
## [1] 8.7
\end{verbatim}

\begin{Shaded}
\begin{Highlighting}[]
\KeywordTok{round}\NormalTok{(((}\DecValTok{6000-5500}\NormalTok{)}\OperatorTok{/}\StringTok{ }\DecValTok{5500}\NormalTok{)}\OperatorTok{*}\DecValTok{100}\NormalTok{,}\DecValTok{2}\NormalTok{)}
\end{Highlighting}
\end{Shaded}

\begin{verbatim}
## [1] 9.09
\end{verbatim}

\end{document}
