% Options for packages loaded elsewhere
\PassOptionsToPackage{unicode}{hyperref}
\PassOptionsToPackage{hyphens}{url}
%
\documentclass[
]{article}
\usepackage{lmodern}
\usepackage{amssymb,amsmath}
\usepackage{ifxetex,ifluatex}
\ifnum 0\ifxetex 1\fi\ifluatex 1\fi=0 % if pdftex
  \usepackage[T1]{fontenc}
  \usepackage[utf8]{inputenc}
  \usepackage{textcomp} % provide euro and other symbols
\else % if luatex or xetex
  \usepackage{unicode-math}
  \defaultfontfeatures{Scale=MatchLowercase}
  \defaultfontfeatures[\rmfamily]{Ligatures=TeX,Scale=1}
\fi
% Use upquote if available, for straight quotes in verbatim environments
\IfFileExists{upquote.sty}{\usepackage{upquote}}{}
\IfFileExists{microtype.sty}{% use microtype if available
  \usepackage[]{microtype}
  \UseMicrotypeSet[protrusion]{basicmath} % disable protrusion for tt fonts
}{}
\makeatletter
\@ifundefined{KOMAClassName}{% if non-KOMA class
  \IfFileExists{parskip.sty}{%
    \usepackage{parskip}
  }{% else
    \setlength{\parindent}{0pt}
    \setlength{\parskip}{6pt plus 2pt minus 1pt}}
}{% if KOMA class
  \KOMAoptions{parskip=half}}
\makeatother
\usepackage{xcolor}
\IfFileExists{xurl.sty}{\usepackage{xurl}}{} % add URL line breaks if available
\IfFileExists{bookmark.sty}{\usepackage{bookmark}}{\usepackage{hyperref}}
\hypersetup{
  hidelinks,
  pdfcreator={LaTeX via pandoc}}
\urlstyle{same} % disable monospaced font for URLs
\usepackage[margin=1in]{geometry}
\usepackage{graphicx,grffile}
\makeatletter
\def\maxwidth{\ifdim\Gin@nat@width>\linewidth\linewidth\else\Gin@nat@width\fi}
\def\maxheight{\ifdim\Gin@nat@height>\textheight\textheight\else\Gin@nat@height\fi}
\makeatother
% Scale images if necessary, so that they will not overflow the page
% margins by default, and it is still possible to overwrite the defaults
% using explicit options in \includegraphics[width, height, ...]{}
\setkeys{Gin}{width=\maxwidth,height=\maxheight,keepaspectratio}
% Set default figure placement to htbp
\makeatletter
\def\fps@figure{htbp}
\makeatother
\setlength{\emergencystretch}{3em} % prevent overfull lines
\providecommand{\tightlist}{%
  \setlength{\itemsep}{0pt}\setlength{\parskip}{0pt}}
\setcounter{secnumdepth}{-\maxdimen} % remove section numbering

\author{}
\date{\vspace{-2.5em}}

\begin{document}

\#NORMALITY TESTS IN ECONOMETRICS USING R\\
\#\#Author: R J Neel\\
\#\#\#Referene: Ben Lambert Econometrics Course, Introductory
Econometrics: A modern approach by J Wooldrige, Econometric Analysis by
W. Greene

Let us consider an example from the \textbf{wage1} dataset. Here, we
test the relationship between wage and education and experience.

\begin{verbatim}
##   wage educ exper tenure nonwhite female married numdep smsa northcen south
## 1 3.10   11     2      0        0      1       0      2    1        0     0
## 2 3.24   12    22      2        0      1       1      3    1        0     0
## 3 3.00   11     2      0        0      0       0      2    0        0     0
## 4 6.00    8    44     28        0      0       1      0    1        0     0
## 5 5.30   12     7      2        0      0       1      1    0        0     0
## 6 8.75   16     9      8        0      0       1      0    1        0     0
##   west construc ndurman trcommpu trade services profserv profocc clerocc
## 1    1        0       0        0     0        0        0       0       0
## 2    1        0       0        0     0        1        0       0       0
## 3    1        0       0        0     1        0        0       0       0
## 4    1        0       0        0     0        0        0       0       1
## 5    1        0       0        0     0        0        0       0       0
## 6    1        0       0        0     0        0        1       1       0
##   servocc    lwage expersq tenursq
## 1       0 1.131402       4       0
## 2       1 1.175573     484       4
## 3       0 1.098612       4       0
## 4       0 1.791759    1936     784
## 5       0 1.667707      49       4
## 6       0 2.169054      81      64
\end{verbatim}

\includegraphics{97_Test_for_Normality_files/figure-latex/unnamed-chunk-1-1.pdf}

We devise the following model:\\
wage=f(education, experience)

Therefore, we devise the following model:\\
\(Wage=\beta_0 + \beta_1*Education +\beta_2*Experience\)

The result of regression is

\begin{verbatim}
## 
## Call:
## lm(formula = wage ~ educ + exper, data = data)
## 
## Coefficients:
## (Intercept)         educ        exper  
##     -3.3905       0.6443       0.0701
\end{verbatim}

\begin{verbatim}
## 
## Call:
## lm(formula = wage ~ educ + exper, data = data)
## 
## Residuals:
##     Min      1Q  Median      3Q     Max 
## -5.5532 -1.9801 -0.7071  1.2030 15.8370 
## 
## Coefficients:
##             Estimate Std. Error t value Pr(>|t|)    
## (Intercept) -3.39054    0.76657  -4.423 1.18e-05 ***
## educ         0.64427    0.05381  11.974  < 2e-16 ***
## exper        0.07010    0.01098   6.385 3.78e-10 ***
## ---
## Signif. codes:  0 '***' 0.001 '**' 0.01 '*' 0.05 '.' 0.1 ' ' 1
## 
## Residual standard error: 3.257 on 523 degrees of freedom
## Multiple R-squared:  0.2252, Adjusted R-squared:  0.2222 
## F-statistic: 75.99 on 2 and 523 DF,  p-value: < 2.2e-16
\end{verbatim}

\begin{verbatim}
## Analysis of Variance Table
## 
## Response: wage
##            Df Sum Sq Mean Sq F value    Pr(>F)    
## educ        1 1179.7 1179.73 111.208 < 2.2e-16 ***
## exper       1  432.5  432.52  40.772  3.78e-10 ***
## Residuals 523 5548.2   10.61                      
## ---
## Signif. codes:  0 '***' 0.001 '**' 0.01 '*' 0.05 '.' 0.1 ' ' 1
\end{verbatim}

Thus, we have \(Wage=-3.39+0.64*Education +0.07*Experience\)

Now, we test for \textbf{NORMALITY}

Let us see the distribution of wages\\
\includegraphics{97_Test_for_Normality_files/figure-latex/unnamed-chunk-3-1.pdf}

Prima facie it looks like the distribution is \textbf{skewed} towards
the right.

Let us conduct all the normality tests

\textbf{{[}1{]}Visual Tests}\\
\includegraphics{97_Test_for_Normality_files/figure-latex/unnamed-chunk-4-1.pdf}

\textbf{{[}2{]}Jarque-Berra Test}

\begin{verbatim}
## Registered S3 method overwritten by 'quantmod':
##   method            from
##   as.zoo.data.frame zoo
\end{verbatim}

\begin{verbatim}
## 
##  Jarque Bera Test
## 
## data:  data$wage
## X-squared = 894.62, df = 2, p-value < 2.2e-16
\end{verbatim}

\textbf{INTERPRETATION}\\
For Jarque-Berra test our\\
\emph{Ho: Distribution is NORMAL}\\
\emph{Ha: Otherwise}\\
A p value of \(<0.05\) implies That the test stastic lies in the
{REJECTION REGION}. Therefore, We \textbf{fail} to conclude that the
distribution is \textbf{NORMAL}. It was already hinted in the Visual
tests.

\textbf{{[}3{]}Shapiro-Wilk Test}

\begin{verbatim}
## 
##  Shapiro-Wilk normality test
## 
## data:  data$wage
## W = 0.80273, p-value < 2.2e-16
\end{verbatim}

As is in the case for Jarque-Berra test, In Shapiro-Wilkins test too =
our\\
\emph{Ho: Distribution is NORMAL}\\
\emph{Ha: Otherwise}\\
A p value of \(<0.05\) implies That the test stastic lies in the
{REJECTION REGION}. Therefore, We \textbf{fail} to conclude that the
distribution is \textbf{NORMAL}. It was already hinted in the Visual
tests.

Shapiro-wilk is the more preferred test.

\textbf{Let us now repeat the exercise with Log(wages)}\\
Let us see the distribution of Log(wages)\\
\includegraphics{97_Test_for_Normality_files/figure-latex/unnamed-chunk-7-1.pdf}

Prima facie it does \textbf{NOT} look \textbf{NORMAL}

Let us conduct all the normality tests

\textbf{{[}1{]}Visual Tests}\\
\includegraphics{97_Test_for_Normality_files/figure-latex/unnamed-chunk-8-1.pdf}

\textbf{{[}2{]}Jarque-Berra Test}

\begin{verbatim}
## 
##  Jarque Bera Test
## 
## data:  log(data$wage)
## X-squared = 16.67, df = 2, p-value = 0.0002399
\end{verbatim}

\textbf{INTERPRETATION}\\
The p value has increased but it is still \(<0.05\) implies That the
test stastic lies in the {REJECTION REGION}. Therefore, We \textbf{fail}
to conclude that the distribution is \textbf{NORMAL}. It was already
hinted in the Visual tests.

\textbf{{[}3{]}Shapiro-Wilk Test}

\begin{verbatim}
## 
##  Shapiro-Wilk normality test
## 
## data:  log(data$wage)
## W = 0.96909, p-value = 4.423e-09
\end{verbatim}

As is in the case for Jarque-Berra test, In Shapiro-Wilkins test too it
fails to prove \textbf{NORMALITY}

\end{document}
