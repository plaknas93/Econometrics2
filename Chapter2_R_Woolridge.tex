\PassOptionsToPackage{unicode=true}{hyperref} % options for packages loaded elsewhere
\PassOptionsToPackage{hyphens}{url}
%
\documentclass[
]{article}
\usepackage{lmodern}
\usepackage{amssymb,amsmath}
\usepackage{ifxetex,ifluatex}
\ifnum 0\ifxetex 1\fi\ifluatex 1\fi=0 % if pdftex
  \usepackage[T1]{fontenc}
  \usepackage[utf8]{inputenc}
  \usepackage{textcomp} % provides euro and other symbols
\else % if luatex or xelatex
  \usepackage{unicode-math}
  \defaultfontfeatures{Scale=MatchLowercase}
  \defaultfontfeatures[\rmfamily]{Ligatures=TeX,Scale=1}
\fi
% use upquote if available, for straight quotes in verbatim environments
\IfFileExists{upquote.sty}{\usepackage{upquote}}{}
\IfFileExists{microtype.sty}{% use microtype if available
  \usepackage[]{microtype}
  \UseMicrotypeSet[protrusion]{basicmath} % disable protrusion for tt fonts
}{}
\makeatletter
\@ifundefined{KOMAClassName}{% if non-KOMA class
  \IfFileExists{parskip.sty}{%
    \usepackage{parskip}
  }{% else
    \setlength{\parindent}{0pt}
    \setlength{\parskip}{6pt plus 2pt minus 1pt}}
}{% if KOMA class
  \KOMAoptions{parskip=half}}
\makeatother
\usepackage{xcolor}
\IfFileExists{xurl.sty}{\usepackage{xurl}}{} % add URL line breaks if available
\IfFileExists{bookmark.sty}{\usepackage{bookmark}}{\usepackage{hyperref}}
\hypersetup{
  pdfborder={0 0 0},
  breaklinks=true}
\urlstyle{same}  % don't use monospace font for urls
\usepackage[margin=1in]{geometry}
\usepackage{graphicx,grffile}
\makeatletter
\def\maxwidth{\ifdim\Gin@nat@width>\linewidth\linewidth\else\Gin@nat@width\fi}
\def\maxheight{\ifdim\Gin@nat@height>\textheight\textheight\else\Gin@nat@height\fi}
\makeatother
% Scale images if necessary, so that they will not overflow the page
% margins by default, and it is still possible to overwrite the defaults
% using explicit options in \includegraphics[width, height, ...]{}
\setkeys{Gin}{width=\maxwidth,height=\maxheight,keepaspectratio}
\setlength{\emergencystretch}{3em}  % prevent overfull lines
\providecommand{\tightlist}{%
  \setlength{\itemsep}{0pt}\setlength{\parskip}{0pt}}
\setcounter{secnumdepth}{-2}
% Redefines (sub)paragraphs to behave more like sections
\ifx\paragraph\undefined\else
  \let\oldparagraph\paragraph
  \renewcommand{\paragraph}[1]{\oldparagraph{#1}\mbox{}}
\fi
\ifx\subparagraph\undefined\else
  \let\oldsubparagraph\subparagraph
  \renewcommand{\subparagraph}[1]{\oldsubparagraph{#1}\mbox{}}
\fi

% set default figure placement to htbp
\makeatletter
\def\fps@figure{htbp}
\makeatother


\author{}
\date{\vspace{-2.5em}}

\begin{document}

\#\textbf{Chapter 2: Simple Linear Regression}

\#\#\emph{Author: J Wooldridge}

\#\#\#\textbf{Regression Practice}

\begin{verbatim}
## 
## Attaching package: 'dplyr'
\end{verbatim}

\begin{verbatim}
## The following objects are masked from 'package:stats':
## 
##     filter, lag
\end{verbatim}

\begin{verbatim}
## The following objects are masked from 'package:base':
## 
##     intersect, setdiff, setequal, union
\end{verbatim}

\#\textbf{Exapmple 2.3 CEO Salary and Return on Equity}

\begin{verbatim}
##    Min. 1st Qu.  Median    Mean 3rd Qu.    Max. 
##     223     736    1039    1281    1407   14822
\end{verbatim}

\begin{verbatim}
##    Min. 1st Qu.  Median    Mean 3rd Qu.    Max. 
##    0.50   12.40   15.50   17.18   20.00   56.30
\end{verbatim}

\begin{verbatim}
## 
## Call:
## lm(formula = salary ~ roe, data = ceosal1)
## 
## Residuals:
##     Min      1Q  Median      3Q     Max 
## -1160.2  -526.0  -254.0   138.8 13499.9 
## 
## Coefficients:
##             Estimate Std. Error t value Pr(>|t|)    
## (Intercept)   963.19     213.24   4.517 1.05e-05 ***
## roe            18.50      11.12   1.663   0.0978 .  
## ---
## Signif. codes:  0 '***' 0.001 '**' 0.01 '*' 0.05 '.' 0.1 ' ' 1
## 
## Residual standard error: 1367 on 207 degrees of freedom
## Multiple R-squared:  0.01319,    Adjusted R-squared:  0.008421 
## F-statistic: 2.767 on 1 and 207 DF,  p-value: 0.09777
\end{verbatim}

\begin{verbatim}
## `geom_smooth()` using formula 'y ~ x'
\end{verbatim}

\includegraphics{Chapter2_R_Woolridge_files/figure-latex/unnamed-chunk-2-1.pdf}

\textbf{Interpretation: }\\
If the ROE of the the firm increases by 1 percentage point than the
CEO's salary increasese by \textbf{18.5} thousand dollars.

\#\textbf{Exapmple 2.4 Wage and Education}

Consider the scatterplot of wage vs education from the wage dataset and
the OLS model estimates

\begin{verbatim}
## 
## Call:
## lm(formula = wage ~ educ, data = wage1)
## 
## Residuals:
##     Min      1Q  Median      3Q     Max 
## -5.3396 -2.1501 -0.9674  1.1921 16.6085 
## 
## Coefficients:
##             Estimate Std. Error t value Pr(>|t|)    
## (Intercept) -0.90485    0.68497  -1.321    0.187    
## educ         0.54136    0.05325  10.167   <2e-16 ***
## ---
## Signif. codes:  0 '***' 0.001 '**' 0.01 '*' 0.05 '.' 0.1 ' ' 1
## 
## Residual standard error: 3.378 on 524 degrees of freedom
## Multiple R-squared:  0.1648, Adjusted R-squared:  0.1632 
## F-statistic: 103.4 on 1 and 524 DF,  p-value: < 2.2e-16
\end{verbatim}

\begin{verbatim}
## 
## Call:
## lm(formula = wage ~ educ, data = wage1)
## 
## Coefficients:
## (Intercept)         educ  
##     -0.9049       0.5414
\end{verbatim}

\begin{verbatim}
## `geom_smooth()` using formula 'y ~ x'
\end{verbatim}

\includegraphics{Chapter2_R_Woolridge_files/figure-latex/unnamed-chunk-3-1.pdf}

\textbf{Interpretation: } 1 more year of education increases wage by a
half a dollar per hour. {NOTE}: The {intercept} value of {-.54} implies
negative hourly wage for no education. This is absurd. A person who has
no education will not be paying his employer for doing work.

\#\textbf{Exapmple 2.5 Voting Outcomes and Campaign Expenditures}

Consider the scatterplot of wage vs education from the wage dataset and
the OLS model estimates

\begin{verbatim}
##   state district democA voteA expendA expendB prtystrA lexpendA lexpendB
## 1    AL        7      1    68 328.296   8.737       41 5.793916 2.167567
## 2    AK        1      0    62 626.377 402.477       60 6.439952 5.997638
## 3    AZ        2      1    73  99.607   3.065       55 4.601233 1.120048
## 4    AZ        3      0    69 319.690  26.281       64 5.767352 3.268846
## 5    AR        3      0    75 159.221  60.054       66 5.070293 4.095244
## 6    AR        4      1    69 570.155  21.393       46 6.345908 3.063064
##     shareA
## 1 97.40767
## 2 60.88104
## 3 97.01476
## 4 92.40370
## 5 72.61247
## 6 96.38355
\end{verbatim}

\begin{verbatim}
## 
## Call:
## lm(formula = voteA ~ shareA, data = vote1)
## 
## Residuals:
##      Min       1Q   Median       3Q      Max 
## -16.8919  -4.0660  -0.1682   3.4965  29.9772 
## 
## Coefficients:
##             Estimate Std. Error t value Pr(>|t|)    
## (Intercept) 26.81221    0.88721   30.22   <2e-16 ***
## shareA       0.46383    0.01454   31.90   <2e-16 ***
## ---
## Signif. codes:  0 '***' 0.001 '**' 0.01 '*' 0.05 '.' 0.1 ' ' 1
## 
## Residual standard error: 6.385 on 171 degrees of freedom
## Multiple R-squared:  0.8561, Adjusted R-squared:  0.8553 
## F-statistic:  1018 on 1 and 171 DF,  p-value: < 2.2e-16
\end{verbatim}

\begin{verbatim}
## 
## Call:
## lm(formula = voteA ~ shareA, data = vote1)
## 
## Coefficients:
## (Intercept)       shareA  
##     26.8122       0.4638
\end{verbatim}

\begin{verbatim}
## `geom_smooth()` using formula 'y ~ x'
\end{verbatim}

\includegraphics{Chapter2_R_Woolridge_files/figure-latex/unnamed-chunk-4-1.pdf}

\textbf{Interpretation: } Vote share increases by almost half a percent
if Cadidate A increases his campaign spending by 1 percent in the total
campaign spending among both.

\#\textbf{Example 2.6 CEO Salary and ROE}

\begin{verbatim}
## [1] 1281.119
\end{verbatim}

\begin{verbatim}
## [1] 1281.12
\end{verbatim}

\begin{verbatim}
## 
## Call:
## lm(formula = salary ~ roe, data = ceosal1)
## 
## Residuals:
##     Min      1Q  Median      3Q     Max 
## -1160.2  -526.0  -254.0   138.8 13499.9 
## 
## Coefficients:
##             Estimate Std. Error t value Pr(>|t|)    
## (Intercept)   963.19     213.24   4.517 1.05e-05 ***
## roe            18.50      11.12   1.663   0.0978 .  
## ---
## Signif. codes:  0 '***' 0.001 '**' 0.01 '*' 0.05 '.' 0.1 ' ' 1
## 
## Residual standard error: 1367 on 207 degrees of freedom
## Multiple R-squared:  0.01319,    Adjusted R-squared:  0.008421 
## F-statistic: 2.767 on 1 and 207 DF,  p-value: 0.09777
\end{verbatim}

\begin{verbatim}
## [1] 0.0121
\end{verbatim}

\includegraphics{Chapter2_R_Woolridge_files/figure-latex/unnamed-chunk-5-1.pdf}

\#\textbf{Example 2.7 Wage and Education}\\
No Analysis required

\#\textbf{Example 2.8 CEO Salary and ROE}\\
The correlation between predicted salary and observed salary is
\textbf{0.11}. This when squared yields \textbf{0.01}.

The results of regression of salary on roe are:

\begin{verbatim}
## 
## Call:
## lm(formula = salary ~ roe, data = ceosal1)
## 
## Residuals:
##     Min      1Q  Median      3Q     Max 
## -1160.2  -526.0  -254.0   138.8 13499.9 
## 
## Coefficients:
##             Estimate Std. Error t value Pr(>|t|)    
## (Intercept)   963.19     213.24   4.517 1.05e-05 ***
## roe            18.50      11.12   1.663   0.0978 .  
## ---
## Signif. codes:  0 '***' 0.001 '**' 0.01 '*' 0.05 '.' 0.1 ' ' 1
## 
## Residual standard error: 1367 on 207 degrees of freedom
## Multiple R-squared:  0.01319,    Adjusted R-squared:  0.008421 
## F-statistic: 2.767 on 1 and 207 DF,  p-value: 0.09777
\end{verbatim}

This explains the plot \texttt{\{r,echo=FALSE\}\ sal\_roe\_plot}

\#\textbf{Example 2.9 Voting Outcomes and Campaign Expenditures}\\
The regresssion outcome as indicated by the plot, cleary, seems to be a
better fit for the model. { But, How much better?}

\begin{verbatim}
## `geom_smooth()` using formula 'y ~ x'
\end{verbatim}

\includegraphics{Chapter2_R_Woolridge_files/figure-latex/unnamed-chunk-7-1.pdf}

We check the R-Square also called the, Coefficient of Determination. In
this case the R square value is \textbf{0.8561409}.

\#\textbf{2-4 Units of Measurement and Functional Form (Page 36)}\\
(1)Change in DEPENDENT VARIABLE UNIT\\
Consider the case of CEO\_Sal regressed on ROE. Here, CEO\_Sal is
interms of thousands of \$ and ROE is measured as percentage. If
CEO\_Sal is instead changed to 100s, what would the impact on beta and
alpha?

The new regression result is as follows:

\begin{verbatim}
##   salary pcsalary   sales  roe pcroe ros indus finance consprod utility
## 1   1095       20 27595.0 14.1 106.4 191     1       0        0       0
## 2   1001       32  9958.0 10.9 -30.6  13     1       0        0       0
## 3   1122        9  6125.9 23.5 -16.3  14     1       0        0       0
## 4    578       -9 16246.0  5.9 -25.7 -21     1       0        0       0
## 5   1368        7 21783.2 13.8  -3.0  56     1       0        0       0
## 6   1145        5  6021.4 20.0   1.0  55     1       0        0       0
##    lsalary    lsales
## 1 6.998509 10.225389
## 2 6.908755  9.206132
## 3 7.022868  8.720281
## 4 6.359574  9.695602
## 5 7.221105  9.988894
## 6 7.043160  8.703075
\end{verbatim}

\begin{verbatim}
##   salary  roe sal_100
## 1   1095 14.1   10950
## 2   1001 10.9   10010
## 3   1122 23.5   11220
## 4    578  5.9    5780
## 5   1368 13.8   13680
## 6   1145 20.0   11450
\end{verbatim}

\begin{verbatim}
## (Intercept)         roe 
##   9631.9134    185.0119
\end{verbatim}

The old regression (CEO\_sal on ROE) result was

\begin{verbatim}
## (Intercept)         roe 
##   963.19134    18.50119
\end{verbatim}

Notice, both intercept and slope get multiplied by \textbf{10}.\\
\textbf{In General, for every multiplication or division by any constant
`c' to the DEPENDENT variable must be repeated across all intercept and
independent variable.}

(2)Change in INDEPENDENT VARIABLE UNIT\\
Consider the case of CEO\_Sal regressed on ROE. Here, CEO\_Sal is
interms of thousands of \$ and ROE is measured as percentage. If ROE is
instead changed to its decimal form, what would the impact on beta and
alpha?

The new regression result is as follows:

\begin{verbatim}
##   salary pcsalary   sales  roe pcroe ros indus finance consprod utility
## 1   1095       20 27595.0 14.1 106.4 191     1       0        0       0
## 2   1001       32  9958.0 10.9 -30.6  13     1       0        0       0
## 3   1122        9  6125.9 23.5 -16.3  14     1       0        0       0
## 4    578       -9 16246.0  5.9 -25.7 -21     1       0        0       0
## 5   1368        7 21783.2 13.8  -3.0  56     1       0        0       0
## 6   1145        5  6021.4 20.0   1.0  55     1       0        0       0
##    lsalary    lsales
## 1 6.998509 10.225389
## 2 6.908755  9.206132
## 3 7.022868  8.720281
## 4 6.359574  9.695602
## 5 7.221105  9.988894
## 6 7.043160  8.703075
\end{verbatim}

\begin{verbatim}
##   salary  roe roe_dec
## 1   1095 14.1   0.141
## 2   1001 10.9   0.109
## 3   1122 23.5   0.235
## 4    578  5.9   0.059
## 5   1368 13.8   0.138
## 6   1145 20.0   0.200
\end{verbatim}

\begin{verbatim}
## (Intercept)     roe_dec 
##    963.1913   1850.1186
\end{verbatim}

The old regression (CEO\_sal on ROE) result was

\begin{verbatim}
## (Intercept)         roe 
##   963.19134    18.50119
\end{verbatim}

Notice, only beta changes and alpha remains unchanged. ROE is multiplied
by 100 for a division of 100 of the independent's units.

\textbf{In General, every multiplication or division by any constant `c'
to the INDEPENDENT variable must be repeated with division or
multiplication resp of across DEPENDENT ONLY.}

\end{document}
